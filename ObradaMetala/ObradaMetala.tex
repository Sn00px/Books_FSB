\documentclass[a4paper,12pt]{article}
\usepackage[croatian]{babel}
\usepackage[utf8]{inputenc}
\usepackage[T1]{fontenc}
\usepackage{gensymb}
\usepackage{tikz}
\usepackage{lscape}
\usepackage{amsmath,amsfonts,amssymb,mathrsfs}
\usepackage{fancyhdr,makeidx}
\usepackage{dcolumn,multirow,eucal,hhline,subcaption}
\usepackage{pgfplots,pgfplotstable,colortbl,array}
\usepackage[unicode, hidelinks]{hyperref}
\usepackage{ragged2e}
\usepackage{
epstopdf,
graphicx,tikz,
%pgflibraryshapes,
color,caption,
listings,
dcolumn,
multirow,
array,
booktabs,
picture,
upgreek,
wrapfig,      
cancel,
placeins,
url,
verbatim,
media9,
float,
incgraph
}
\usepackage[nottoc,numbib]{tocbibind}
\usepackage[croatian]{nomencl}
\makenomenclature
\usepgfplotslibrary{units}
\usetikzlibrary{pgfplots.units}
\usetikzlibrary{angles,calc,decorations.pathmorphing,patterns}
\usetikzlibrary{decorations.pathmorphing,decorations.pathreplacing}
\pgfplotstableset{precision=10,set thousands separator={}}
\usepackage{nicefrac}
\hoffset -30 pt
\voffset -50 pt
\textheight = 650 pt
\textwidth = 450 pt
\sloppy
\definecolor{lightgray}{gray}{0.5}
\setlength{\parindent}{0pt}
\usepgfplotslibrary{external}
\usetikzlibrary{pgfplots.external}
\usepgfplotslibrary{external}
\usepackage{lmodern,textcomp}
\usepackage{multimedia}
\headheight = 14 pt
\headwidth = 17 cm
\pagestyle{fancy}


\newcommand{\tikzAngleOfLine}{\tikz@AngleOfLine} %crtanje kuteva
  \def\tikz@AngleOfLine(#1)(#2)#3{%
  \pgfmathanglebetweenpoints{%
    \pgfpointanchor{#1}{center}}{%
    \pgfpointanchor{#2}{center}}
  \pgfmathsetmacro{#3}{\pgfmathresult}%
  }


\fancyfoot[R]{\thepage}
\fancypagestyle{fancypage}{
    \fancyhf{}
    \footrulewidth=20pt
    %\renewcommand{\headrulewidth}{0pt}
    \renewcommand{\footrulewidth}{0.4pt}
}
\cfoot[]{}
\renewcommand{\thefigure}{\arabic{section}.\arabic{subsection}.\arabic{figure}}
\makeindex
\numberwithin{figure}{section}

\setlength\parindent{24pt}
\begin{document}

\begin{titlepage}
\begin{titlepage}
  

  \begin{center}

  {\huge\bfseries Obrada metala}
  \vskip 2cm
  
  \end{center}

\end{titlepage}
\end{titlepage}

\section{Teorija rezanja metala}
\subsection{Alati za rad skidanjem strugotine}
Obrađivanje skidanjem strugotine, kao np.: rezanje, tokarenje, struganje, glodanje, brušenje, provlačenje, grecanje, vrši se raznim alatima. Jednom to može biti \textbf{nož} s jednom oštricom (tokarenje, blanjanje), drugi puta \textbf{svrdlo} koje na sebi nosi dvije osnovne oštrice, treči puta alat sa više oštrica koje odjednom skidaju strugotinu mora, kao i ostali zahvatiti, odgovarati nekim zakonitostima. Te zakonitosti prvi je počeo proučavati amerikanac F.W.Taylor krajem prošlog stoljeća.\par
Svaki alat, koji koristimo pri skidanju strugotina, kao što smo vidjeli, sastoji se of jedne ili više oštrica. To nas dovodi do zaključka, da analogijom tu zakonitost prenesemo na sve druge.\par
Osnovna oštrica alata za rezanje bazira se na principu klina. Znači, moramo proučiti djelovanje klina. Zbog toga mi ćemo se zadržati na jednom alatu, a to je \textbf{tokarski nož}, te ćemo na njemu proučiti sve.
\subsection{Geometrijski oblici oštrice alata i elementi oštrine}
Rekli smo da je temeljni oblik svakog alata za rezanje klin. To vrijedi kod dljeta, kod oštrice škara, kod turpije, tokarskog noža, glodala itd. \par
Prvo da se upoznamo opčenito s tokarskim nožem i njegovim elementima rezanja i plohama.

%SLIKA TOKARSKOG NOŽA ||||| A1
Osnovni elementi su:
\begin{itemize}
\item glavna oštrica
\item pomoćna oštrica
\item noža
\end{itemize}
Od površina koje trebamo razlikovati na nožu su
\begin{itemize}
\item prednja površina
\item stražnja površina
\item pomočna stražnja površina
\end{itemize}
Prije nego što upoznamo detalje noža, spomenimo i sile koje se javljaju na nožu. To su:
\begin{itemize}
\item $\mathsf{P}$ - vertikalna sila- sila protivna rezanju
\item $\mathsf{P_{v}}$ - sila posmaka - sila protivna uzdužnom posmaku
\item $\mathsf{P_{r}}$ - natražni pritisak - sila protivna poprečnom posmaku, a nastoji otisnuti nož od predmeta
\end{itemize}
Sada ćemo se upoznati malo detaljnije s jednim nožem za tokarenje, nožem koji se najčešće koristi.

%SLIKE KUTEVA REZANJA TOKARSKOG NOŽA ||||| A4

Analizirajući presjek x-x razlikujemo:
\begin{itemize}
\item $\alpha$ - stražnji ili slobodni kut - između stražnje površine i površine rezanja. 
\item $\beta$ - kut oštrenja ili kut klina - kut između prednje i stražnje površine.
\item $\gamma$ - prednji kut, kut između prednje površine i okomice na površinu rezanja. Proizlazi da je $\alpha$ + $\beta$ + $\gamma$ = $90^{\circ}$
\item $\delta$ - kut rezanja - on je suma $\alpha$ + $\beta$.
\item $\epsilon$ - kut šiljka noža ili čeoni kut, to je kut između stražnje i pomočne stražnje površine.
\item $\kappa$ - kut namještanja noža je kut između površine rezanja i glavne oštrice.
\item $\lambda$ - kut nadvišenja - kut između glavne oštrice i površine u ravnini rezanja koja proizlazi kroz vrh noža.
\item $\tau$ - natražni kut - koji se pojavljuje kod noževa za odrezivanje.
\end{itemize}
Evo nekoliko osnovnih primjera brušenja tokarskih noževa.

%SLIKE BRUŠENJA TOKARSKIH NOŽEVA |||| A/5

Položaj noža prema tokarenom predmetu može također djelovati na promjene osnovnih kuteva rezanja $\alpha$, $\beta$, $\delta$, dok nam kut $\beta$ ostaje konstantan


%SLIKA KONSTANTNOG KUTA BETA |||| A/6


Kod unutarnjeg tokarenja situacija je obratna. To nas upučuje da je važno paziti kako je postavlja nož kod obrade. 
\subsection{Proces rezanja i formiranja strugotine}
Prilikom skidanja strugotine sudjeluje niz faktora koji definiraju oblik strugotine. U te faktore možemo svrstati:
\begin{itemize}
\item vrstu materijala koji se obrađuje
\item vrstu materijala od kojeg je nož
\item brzinu rezanja i 
\item presjek strugotina
\end{itemize}
Postoje tri osnovna tipa strugotine:
\begin{enumerate}
%SLIKA
\item \textbf{Kidana - lomljena strugotina}\\
Materijal se najprije sabije na prednjoj površini noža, a zatim se zbog povečanog pritiska odlomi. Nakon toga proces se opet ponavlja. \\
Kidana - lomljena strugotina nastaje kada je prednji kut mani od $10^{\circ}$, a zatim kod tvrdih materijala i kada se radi s preniskim brzinama rezanja. U tako formiranoj strugotini ne nastaju plastične deformacije. Oštrica noža se grije do $600^{\circ}$C i to nejednoliko. Kako se pritisak mijenja, mijenja se i zagrijavanje oštrice što dovodi do velikog kolebanja temperature. Kod takovog obnlika strugotine podmazivanje noža mnogo ne pomaže. Takav oblik strugotine javlja se kod ljevanjog željeza.
%SLIKA
\item \textbf{Rezana ili odrezna strugotina}\\
Ovakav oblik strugotine nastaje kod prednjeg kuta $\gamma < 17^{\circ}$ i kod male dubine rezanja. Taj oblik strugotine je prijelazni oblik od kidane na trakastu strugotinu. Oblik strugotine je povoljan jer nije preduga i ne smeta pri radu. 
%SLIKA
\item \textbf{Trakasta ili ljuštena strugotina}\\
Ovakav oblik strugotine nastaje kod velikih brzina rezanja, male dubine i malog posmaka i kod prednjeg kuta $\gamma<30^{\circ}$. Materijal se tari o prednju površinu noža i odlazi kao neprekinuta strugotina. Pritisak na nož je jednolikiji, što daje i jednoličniju temperaturu oštrice od približno $200^{\circ}$C. Kolebanje temperature noža su u granicama od $20^{\circ}$. dobrim podmazivanjem oštrice pri rezanju može se produžiti vijek rada noža, jer takovo podmazivanje pogoduje stvaranju povišene oštrice koja štedi oštricu. To svojstvo nam veoma važno kod rada na automatima.
\end{enumerate}
\subsection{Hrapavost obrađene površine}
Kvaliteta obrađene površine ovisi o režimima rada i o izboru noža. Razlikujemo dvije osnovne forme noževa i to:
\begin{itemize}
\item noževi za grubu obradu
\item noževi za finu obradu
\end{itemize}
Noževi za grubu obradu moraju biti izabrani tako da omoguće maksimalno moguće skidanje strugotina. Sa takovim noževima možemo postići prosječnu hrapavost  $9 - 11\:\mu m$. Noževi za finu obradu već imaju posebne oblike raznih oštrica, te uz dobro izabrane režime rada mogu dati kvalitetu površine of $6-9 \mu m$. Općenito uzevši, tokarski nož za vanjsku obradu možemo prvi moment podijeliti na lijeve i desne noževe. Dali je nož lijevi ili desni oderđujemo na sljedeći način. Uzmemo ga u ruku tako, daoštrica bude okrenuta prema našem tijelu i to prema gore. Strana na kojoj se nalazi glavna rezna oštrica vrijedi kao oznaka (lijevi, desni).\par
Druga podjela noževa za tokarenje može biti:
\begin{itemize}
\item noževi za vanjsko tokarenje.
\item noževi za unutarnje tokarenje
\item razni fazonski noževi za vanjsko ili unutarnje tokarenje a tu spadaju i noževi za rezanje raznih vrsta nareza, raznih oblika utora itd.
\end{itemize}
Pregled noževa za vanjsko tokarenje
%SLIKA
\begin{enumerate}
\item - desni savijeni nož za čeono grubo obrađivanje
\item - desni savijeni nož za čeono obrađivanje uglova
\item - desni ravni nož za uzdužno grubo obrađivanje
\item - desni savijeni nož za uzdužno grubo obrađivanje
\item - nož za fino obrađivanje, šiljati
\item - nož za fino obrađivanje
\item - nož za utore
\item - nož za odrezivanje
\item - ravni nož za poprečno tokarenje
\end{enumerate}
Već smo više puta spominjali termine \textbf{dubina}, \textbf{posmak}, \textbf{presjek strugotine}. Svi ovi termini, veoma se ćesto koriste, kada se govori o skidanju strugotina. Da se sada upoznamo i s tim tako važnim podatcima.
%SLIKA
\begin{itemize}
\item Dubina rezanja - t - je dubina prodiranja noža u materijal i mjeri se u milimetrima
\item Posmak noža - s - mm/okretaj - to je pomak noža duž osi obrađivanog predmeta za svaki okretaj
\item Presjek strugotine - f - mm$^2$ - možemo ga smatrati umnoškom posmaka \textbf{s} i dubine rezanja \textbf{t}; to bi bio teoretski presjek. Stvarni presjek je manji za neskinuti ostatak, koji ovisi o posmaku noža, o kutevima glavne i sporedne oštrice, te o zaobljenosti vrha noža.
\end{itemize}
Analizirajući sliku, veličina neskinutog ostatka naglo raste s porastom posmaka, a isto tako je vidljivo da naglo pada s porastom zaobljenja vrha noža. Jedan od oblika noža koji ostavlja veoma mali ostatak je oblik koji predlaže Taylor. Zbog velike zaobljenosti vrha ostatak je minimalan. No jako zaobljeni noževi imaju cca $15\%$ veće sile rezanja od običnih noževa.
%SLIKA
Specifična opterećenja duž ovih oštrica su dosta nejednolika. Zbog takove forme otežana je proizvodnja takovih noževa i međufazno prebrušavanje što dovodi do rijeđe primjene ovakovih oblika noževa. 
\subsubsection*{Sile na nožu}
U početku smo se već upoznali sa silama, koje se javljaju na nožu. Kod toga razlikujemo:
\begin{itemize}
\item $\mathsf{P}$ - vertikalnu silu - silu protivnu rezanju
\item $\mathsf{P_{v}}$ - sila posmaka - sila protivna uzdužnom posmaku
\item $\mathsf{P_{r}}$ - natražni pritisak - sila protivna poprečnom posmaku nastoji otisnuti nož od predmeta
\end{itemize}
Za nas je najinteresantnija vertikalna sila $\mathsf{P}$, jer je ona znatno veća od drugih dviju sila. Odprilike možemo uzeti da je omjer sila na nožu $\mathsf{P}$ : $\mathsf{P_{v}}$ : $\mathsf{P_{r}}$ = $5$ : $2$ : $1$. Iz toga je vidljivo da nam dimenzioniranje drška noža dovoljno uzeti samo vertikalnu silu $\mathsf{P}$. \par
Na veličinu vertikalne sile djeluju mnogi faktori. Kao najutjecajnije smatramo čvrstoću obrađivanog materijala i presjek strugotine. Povezanost ovih dviju varijabli prikazat ćemo u dijagramu.
%DIJAGRAM
Takova zakonitost sile $\mathsf{P}$ navodi nas da ju možemo brzo i jednostavno odrediti formulom:
\begin{equation}
\mathsf{P} = A \cdot f \cdot \sigma_{z}\:.
\end{equation}
Gdje je $A$ faktor proporcionalnosti ovisan o materijalu, $f$ presjek strugotine ($s\cdot t$)  i $\sigma_{z}$ čvrstoća materijala na vlak. \par
Eksperimentalno su dobiveni podatci za faktor proporcionalnosti i on se kreće:
\begin{itemize}
\item A = $2,5 - 3,5$ za čelične materijale
\item A = $4,5 - 5,5$ za lijevano željezo
\item A = $3 - 4$ za A8 i A8-legure.
\end{itemize} 
Na veličinu sile rezanja ne utjeće samo čvrstoća materijala i presjek strugotine nego još i
\begin{itemize}
\item oblik presjeka strugotine koju može imati različite omjere posmaka i dubine rezanja, a može i oštrica biti zaobljena pa je presjek skidane strugotine duž oštrice raznolik.
\item o kutu klina - oštrenja $\beta$
\item o kutui namještanja noža $\kappa$
\item o brzini
\item o hlađenju i mazanju - kod lomljene strugotine sistemom mazanja nemožemo smanjiti silu.
\item o unutaranjim naprezanjima u materijalu koji obrađujemo.
\end{itemize} 
Snaga potrebna za stvaranje strugotine, troši se najvećim dijelom na rada deformacije strugotine ($75\%$), zatim na rad rezanja ($15\%$) i na rad trenja ($10\%$).
Da bi sve ove faktore uzeli u obzir pri određivanju sile rezanja koristimo se formulom:
\begin{equation}
\mathsf{P} = f \cdot k_{s}\:,
\end{equation}
gdje je $f$ presjek strugotine, a $k_{s}$ koeficijent otpora rezanja. Koeficijent $k_{s}$ određuje se eksperimentalno ili približno prema nekim formulama.
Za točno definiranje sile rezanja pri nekim uvjetima uz odabrane režime i oblik noža, moguće je jedino saznati kroz eksperimente.
\subsubsection{Radnja rezanja}
Kod uzdužnog tokarenja, od triju sila koje se pojavljuju na nožu potrebno je savladati vertikalnu silu $\mathsf{P}$ i horizontalnu silu posmaka $\mathsf{P_{v}}$. Da bi odredili radnju rezanja uz poznavanje tih sila trebamo znati i brzinu rezanja $v$ i brzinu posmaka $v_{s}$ (m/min). Iz tih podataka dobivamo
\begin{equation}
N_{rez} = \mathsf{P} \cdot \frac{v}{60 \cdot 75} + \mathsf{P_{r}} \cdot \frac{v_{s}}{60 \cdot 75}
\end{equation}
Pošto je $\mathsf{P_{r}}$ 2 do 3 puta manja od $\mathsf{P}$ a $v_{s}$ i preko 100 puta manja od $v$, možemo drugi član ove formule zanemariti, te nam preostaje
\begin{equation}
N_{rez} = \frac{\mathsf{P} \cdot v}{4500}
\end{equation}
Da bio saznali potrebnu snagu za vršenje tokarenja, potrebno je uz radnju savladati i ostale otpore koji se javljau u tokarskom stroju. Mjerenja su pokazala da je ukupna potrebna snaga stroja za $50\%$ veća od radnje rezanja što nam daje 
\begin{equation}
N_{tot} =1.5 \cdot N_{rez} = \frac{\mathsf{P} \cdot v}{3000}
\end{equation}
\subsubsection{Mjerenje sila na noževima}
Prva podjela uređaja kojima mjerimo sile na nožu su na mehaničke uređaje i elektroničke metode. Mehanički uređaji za mjerenje sile sastoje se od ploče, poluge ili pera, na koji djeluje sila. Zbog djelovanja vanjske sile dolazi do elastične deformacije odnosno do pomaka iz nultog položaja. Veličina pomaka mjerimo raznim instrumentima (komparatorima ili manometrima) koji su baždareni da pokazuju veličinu sile.\par
Elektroničke metode registriranja mogu biti ostvarene na više načina:
\begin{enumerate}
\item \textbf{Piezoelektroničkim} - prijenos sile kvarcne kristale u kojima se javlja elektrostatski naboj čiju količinu mjerimo. Veličina elektrostatskog naboja proporcionalna je veličini djelujuće sile na kristale.
\item \textbf{Kapacitivnim} - djelovanje vanjske sile izaziva pomake koji se prenose neki elastični ili cilindrični kondenzator. Kondenzator je uključen u krug struje, a njegovom deformacijom mijenja se jakost struje čiju promjenu registrira galvanometar, baždaren u tu svrhu kilogramima.
\item \textbf{Induktivni} - pod djelovanjem vanjske sile dolazi do promjene zračnog raspora a to mijenja jakost inducirane struje u sekundarnoj zavojnici. Mjerenjima te promjene dolazimo do veličine sile koja je izazvala promjenu zračnog raspora.
\item \textbf{Magnetoelastični} - način se temelji na jakom utjecaju mehaničkih sila na krivulju magnetiziranja magnetoelastičnih materijala. Kroz svitak žice teće izmjenična struja, čija jakost ovisi o priključenom naponu, električnom otporu žice a i promjene megnetskih svojstava jezgre djeluju na veličinu struje čije promjene registriramo na instrumentima.
\item \textbf{Elektrolitički način} - električna vodljivost nekog elektrolita ovisi o specifičnom otporu samog elektrolita i o presjeku sloja elektrolita kroz koji prolazi struja. Vanjska sila mijenja presjek sloja, vodljivog elektrolita a time povećava njegov otpor koja protiće elektrolitom. Ampermetar registrira promjenu jakosti struje koja je neka funkcija sile.
\item \textbf{Elektrootporni način} - iz fine žice velikog otpora ($100 - 1000\:\ohm$) napravljene su mjerne trake. Ljepljenjem tih traka na mjesta koja će preživjeti deformacije zbog djelovanja vanjskih sila moći ćemo saznati veličinu sile mjereći promjene jakosti struje koja protiće kroz mjernu traku. Zbog deformacija predmeta na koje su zaljepljene mjerne trake deformirat će se žica u mjernim trakama, što opet izaziva promjenu otpora žice koju registrira instrument. 
\end{enumerate}
Danas u principu najviše koriste metode električnog mjerenja jer imaju niz  prednosti a najvažnija je jednostavnost.
\subsection{Toplina pri procesu rezanja}
Nastanak topline pri procesu skidanja strugotine je višestruka:
\begin{enumerate}
\item \textbf{Mehanički rad} - pretvara se djelomično u toplinu zbog međusobnog trenja čestica strugotina, kod formiranja strugotine i zbog kidanja strugotine.
\item \textbf{Toplina koja se stvara zbog trenjastrugotine o oštricu noža} - porizvedena toplina ovisi o 
\begin{itemize}
\item brzini rezanja
\item debljini strugotine
\item materijala koji se obrađuje
\end{itemize}
%SLIKA
\item \textbf{Strugotina odvodi} sa sobom količinu topline koja proporcionalna presjeku strugotine. Jasno da tu vezano odvođenje topline i s brzinom rezanja.
\item \textbf{Nož odvodi} u principu uvijek iztu količinu topline, i ima zbog svojih dimenzija, konstantni toplinski kapacitet.
\item \textbf{Obrađeni predmet} se zbog preuzetog dijela topline obično rastegne, odnsono promjeni svoje dimenzije. Zbog toga se predmet mora mjeriti na temperaturi okoline.
\end{enumerate}
Toplina koja se prenosi na nož i diže temperaturu noža nepovoljno nam djeluje na izdržljivost oštrice.
%DIJAGRAM
Iz dijagrama je vidljivo da je izdržljivost oštrice, tj. vrijeme rezanja u minutama, manje čim je temperature oštrice veća.\par 
Za razne tipovealatnih strojeva, poželjno nam je da nam neki nož izdrži određeno vrijeme rada s nekim optimalnim režimima. Ti zahtjevi su inicirali najrazličitije zahvate za pronalaženje materijala koji će moći izdržati sve teže režime rada. to je dovelo do odkrivanja materijala koji posjeduju velike mogužnosti i kod vrlo visokih temperatura. Tako možemo ukratko reći da su najpoznatiji materijali iz kojih se rade alati za skidanje strugotina:
%SLIKA
\begin{enumerate}
\item \textbf{Alatni čelici} - osnovna im je karakteristika da u sebi imaju $1-1,45\:\%$ ugljika kad se na $700 - 850^{\circ}$C a napuštaju kad dođu na $180 - 250^{\circ}$C. Noževi od alatnog čelika se lako i dobro obrađuju a daje da su jeftiniji.
\item \textbf{Brzorezni čelici} - su legirani čelici koji u sebi sadrže volframa (W), kroma (Cr), vanadija (V) i molibdena (Mo). Kod kaljena prvo se polagano grije do crvenog žara a onda naglo do $1250 - 1350^{\circ}$C. Napušta se na temperaturu $200 - 270^{\circ}$C ili $575 - 600^{\circ}$C što ovisi o postotcima legiranih elemenata.
\item \textbf{Stelit} - je us tvarnosti tvrđa slitina koja omogućava korištenje još većih brzina rezanja od brzoreznih čelika. Stelit ima približno sastav od $40-50\:\%$ Co, $25 - 35\:\%$ Cr, $12 - 20\:\%$ W, $0,5-3\:\%$ C.
\item \textbf{Widia} - (wie Diamant) - materijal koji omogučava daljnje povečanje režima rada na alatnim strojevima. Za razliku od stelita koji se dobiva ljevanjem, widia se dobiva tako da se prašak volframova titana i molibdena s nekim vezivnim sredstvom (običnom kobaltom ili niklom) veže. Takvoj smjesi se prešanjem daje određeni oblik i zati se peće. Jedno pećenje je na $1400^{\circ}$C a drugo na $1900^{\circ}$C. Obrađivati pločice možemo jedino brušenjem.
\item \textbf{Dijamanti} - noževi napravljeni od dijamanata su noževi koji omogućavaju maksimalne režime rada. Korištenje dijamantnih noževa je opravdano kod fine obrade slitina, papira, gume, umjetinih masa bronce i mesinga. Ne siplati se koristiti za nekaljene čelike. Možemo približno uzeti da je odnos cijena između brzorezanog čelika, trvde slitine i dijamanta 1 : 5 :150. Omjer brzine rezanja je 1 : 6: 10, a dozvoljene temperature rezanja 1 : 1,35 : 3.\par
Na kraju izlaganja o temperaturama kod skidanja strugotina moramo još spomenuti da postoji čitav niz rashladnih sustava koja jednom odvode dio topline a kroz to omogučavaju povećanje režima rada. Uz odvođenje topline rashladno sredstvo omogučava dobivanje glatke površine nakon obrade, pospješujeodvođenje strugotine. Kao najpoznatije rashladno sredstvo je \textbf{emulzija} mlječna otopina sapuna i mineralnog ulja. Osim toga koriste se razne vrste ulja koja smanjuju trenje, dok se kod obrade ljevanog željeza za hlađenje noža koristi komprimirani zrak koji ujedno odpuhuje lomljenu strugotinu.
\end{enumerate}
\subsection{Ekonomske brzine rezanja}
Kod skidanja strugotina poželjno nam je da nam brzina rezanja bude što veća, jer nam to omogućava produktivniju proizvodnju a kroz to i ekonomičniju proizvodnju. Velike brzine rezanja, izazivaju brže zatupljenje noža. pa je potrebno češće prebrušavanje i ponovno namještanje noža. \par
Povezanost troškova obrade i troškova namještanje alata u ovisnosti brzine rezanja možemo prikazati u dijagramu.
%DIJAGRAM
Iz dijagrama je vidljivo da postoji samo jedna ekonomska brzina rezanja za neki alat. Tako se za to koriste noževi od brzorezanog čelika. Ekonomska brzina smatra onu brzinu rezanja kod koje alat između dva prebrušavanja izdrži 60 okr/min, a označava se s $^{\nu}$60.\par 
Za složenije alate, za koje je vrijeme prebrušavanja znatno veće a često prebrušavanje smanjuje viejk trajanja alata odabiru se kao ekonomske brzine $^{\nu}$240 i $^{\nu}$480. U tu grupu alata ulaze fazonski noževi, glodala i drugi. \par
I kod jednostavnijih alata koji se doduše brzo prebrušpavanju ali je njegovo namještanje skopčano s većim gubitkom vremena oko ponovnog namještanja na stroj, često se koriste brzine $^{\nu}$480. to se odnosi na alate koji se koriste na automatskim strojevima. \par
Kod svrdala se ekonomskom brzinom smatra ona brzina, kod koje svrdlo može izbušiti ukupno 2000 mm rupa. Ta se brzina označava s $^{\nu_{l}}$2000.
\section{Blanjanje i dubljenje}
Blanjanje je obrada predmeta skidanjem strugotina s jednim nožem. Međusobno pomicanje noža i predmeta je po pravcu. Kod toga razlikujemo dvije mogučnosti međusobnog gibanja noža i predmeta:
\begin{enumerate}
\item Nož vrši glavno kretanje a predmet vrši posmak. Takav način rada primjenjuje se kod kratkohodnih blanjalica ili šepinga. Maksimalna dužina hoda noža je fo 700 mm.
%SLIKA
\item Predmet vrši glavno kretanje a nož vrši posmak. Predmet je učvršćen na stolu koji se gibapo saonicama, a nož je učvršćen na zasebnoj gredi po kojoj ostvarije posmak. Takav naćin međusobnog gibanja noža i predmeta primjenjuje se kod dugohodnih blanjalica, a točnost obrade je bolje nego kod kratkohodnih blanjalica.
%SLIKA
\end{enumerate}
Kod glavnog kretanja blanjalica razlikujemo:
\begin{itemize}
\item Radni hod - kad anož skida strugotinu
\item Jalovi hod - kada se nož vrača
\end{itemize}
U grupu blanjlica spada i vertikalna blanjalica, od koje je gibanje noža vertikalno i to
\begin{itemize}
\item odozgo na dolje - glavno kretanje - radni hod
\item odozdo na gore - glavno kretanje - jalovi hod
\item posmak stola može biti u ravnini ili po luku.
\end{itemize}
Postoje i specijalne vrste blanjalica za izradu zupčanika. Kao najpoznatije tehnologije blanjalica zupčanika su
\begin{itemize}
\item FELLOW postupak - gdje je alat u obliku zupčanika.
\item Maagov postupak - gdje je alat u obliku zubne letve.
\end{itemize}
\subsection{Kratkohodna blanjalica - šeping}
Jedan od klasičnih strojeva za obradu malih predmeta, koji ne traže naročito kvalitetne površine je šeping. Razlikujemo dvije vrste pogonskih uređaja za ostvarivanje glavnog kretanja pri šepingu i to:
\begin{itemize}
\item pogon na ekscentar
\item hidraulički pogon
\end{itemize}
Princip rada pepinga s pogonom na ekscentar. Pogon je  ostvaren preko elektromotora, remenice i zupčanika do kulisnog mehanizma. Na tom mehanizmu možemo ostvariti veči ili manji ekscentritet kulisnog kamena. Na taj način dobivamo veći ili kraći pomak glave na koju je učveršćen nož. Područje pomicanja glave može se posebno regulirati. Posmak predmeta je osiguran preko drugog kulisnog mehanizma poji je povezan s glavnim.
%SILKA
Nosač noža može ostvariti vertikalno ručno pomicanje, a može se i nagnuti pod izvjesnim kutom za obradu kosina. Radni stol, na kome je smješten predmet može se pomicati ne samo horizontalno, kao nastajanje posmaka, nego i vertikalno pomoću zupčastog prijenosa. Zbog svojeg tereta stol se obično ukruti posebnim podpornjem. 
%SLIKA
Dijagram brzina kod kulisnog mehanizma izgleda ovako. Kulisa se kreće od jedne tangente na kružnicu do druge i onda nazad. Na taj način ostvaruje dužina hoda L. Zbog konstantne kutne brzine $\omega$, brzine gibanja glave šepinga od 1 prema 2 po kutu $\alpha$ raste od 0 do $Vr_{max}$. Dok na povratku od 2 na 1 po kutu $\beta$ raste do $Vp_{max}$ i onda pada na nulu. Zbog konstantne kutne brzine i zbog raznolikosti kuteva $\alpha$ i $\beta$ vrijedi odnos:
\begin{equation}
\frac{\mathsf{tr(vrijeme rad.\:hoda)}}{tp(vrijeme par.\:hoda)} = \frac{V_{ps}}{V_{rs}} = \frac{\alpha}{\beta}>1
\end{equation}
Iz tih podataka, uz poznavanje broja okretaja kulise i dužine puta možemo odrediti brzine rezanja. Stalna promjena brzine rezanja nepovoljno djeluje na kvalitet obrađene površine i na sile na nožu, gdje se javljaju udarci. Da bi se izbjeglo to stalno mijenjanje brzine kod pomicanja noža konstatuirane su blanjalice na hidraulički pogon, kod kojih je brzina radnog ili povratnog hoda, duž čitave radne dužine jednaka. (HRIBAR - 168 strana)
\subsection{Dugohodne blanjalice}
Kod dugohodnih blanjalica, brzina vraćanja radnog stola je  za $2 -3,5$ puta veća od radne brzine. Dugohodne blanjalice mogu raditi i s više od jednog noža što ovisi o veličini blanjalice. Kretanje radnog stola ostvarljivo je ili mehanički preko sistema zupčanika ili hidraulički. Kod manjih dugohodnih blanjalica pogon za automatski posmak noža preuzima se od glavnog pogonskog motora, dok kod velikih dugohodnih blanjalica automatski posmatk noža ima svoj zaseban prigon. Učvršćenje alata izvodi se na glavi predviđenoj za tu svrhu.
%SLIKA
Specifičnost ovog načina stezanja noža je u tome što glava omogučava podizanje noža od površine pri povratnom hodu. To odmicanje je potrebno, da nož ne zapne za obrađenu površinu i ne ošteti se.\par 
Ako za rad na blanjalici koristimo nož za tokarenje, djelovanje sila na nožu mogu izazvati nepoželjnu deformaciju noža koja može imati višestruke posljedice. \par
Nož upet na ovaj naćin, pod djelovanjem sile rezanja opterećenje je na savijanje. Točka oko koje se javlja moment savijanja je nula. Zbog djelovanja tog momenta, nož će preživjeti izvjesnu elastičnu deformaciju, što izaziva vertikalni pomak oštrice prema dolje za visinu e. Zbog nehomogenosti alata, veličina sile (momenta) se mijenja, usljed čega se mijenja i "e" a kroz to dobivamo lošu površinu i velika je vjerojatnost da će nam vrh noža puknuti. Zbog toga su noževi za blanjanje posebnog oblika, da bi se izbjegle što je moguće više te neugodnosti. Noževi za blanjanje su zbog toga savinuti.
%SLIKA
Kod noževa za blanjanje razlikujemo:
\begin{itemize}
\item noževe za grubu obradu
\item noževi za finu obradu
\item noževi za prostornu obradu
\end{itemize}
Za učvršćenje noževa postoje i posebne glave koije omogučavaju stezanje noževa i pod kutem. Za blanjanje se koriste i fazonski noževi za obradu np.: utora, vanjskih zakrivljenja itd.
%SLIKE!
\section{Tokarenje}
Tokarski strojevi su strojevi koji spadaju u najpotrebnije strojeve bilo kakove metaloprerađivačke industrije. Oni posjeduju veliku unirezalnost jer je na njima moguće obrađivati: ravne i zakrivljene površine, mogu se rezati narezi, bušiti i zakrivljene površine, mogu se rezati plohe, razvrtati, irzađivati spiralne opruge, glodati horizontalne utore itd. Tokarski strojevi su veoma jednostavni a predmet koji obrađujemo dobro se vidi. Alati su jednostavni, jeftini i mogu se brzo i lako postavljati i skidati.\par
Kod tokarskih strojeva glavno kretanje vrši predmet i to rotira. Posmak vrši nož, a tokarski stroj obično ima samo jedan. Da bi se povečala produktivnost stroja, a posebno kod velikih tokarski strojeva obrada se može vršiti i s dva ili više noževa. No takove mogućnosti, strojevi imaju po dva suporta a katkada se na jednom suportu, ali s dvije strane postavljaju noževi.\par 
Mane tokarskih strojeva su:
\begin{itemize}
\item obrada se često vrši samo jednom oštricom i zato je potrebno često prebrušavanje.
\item ako je predmet nepravilnog oblika njegovo učvršćivanje na stroj je nezgodno.
\item otežano centriranje predmeta što traži više vremena pa je skupo
\item stroj zauzima relativno mnogo mjesta.
\end{itemize}
Prednosti tokarskog stroja su:
\begin{itemize}
\item jednostavno rukavanje strojem.
\item jednostavni i jeftini alati
\item preglednost obrade
\item univerzalnost obrade koju smo već prije spomenuli.
\end{itemize}
Opčenito gledajući, svi tokarski strojevi su jednaki. Jedan od drugoga se razlikuju po broju radnih brzina i posmaka, veličinom instalirane snage i masivnošć u. To vrijedi za obične tokarske strojeve. Za specijalne svrhe razvijen je čitav niz tokarskih strojeva. \par
Sve to nas upučuje na to da tokarske strojeve podijelimo na:
\begin{enumerate}
\item Jednostavne tokarske strojeve\\
Koji mogu biti:
\begin{itemize}
\item Tokarski strojevi s nareznim vretenom
\item Tokarski strojevi s poteznim vretenom
\item Tokarski strojevi s poteznim i nareznim vretenom.
\end{itemize}
\item Specijalni tokarski strojevi\\
S obzirom na specijalnost oni se dijele na:
\begin{itemize}
\item Obične specijalne tokarske strojeve
\begin{itemize}
\item za čeona tokarenja - čeoni tokarski stroj (301 str.)
\item Korusel tokarski stroj (302 str.)
\item tokarski stroj za profilno tokarenje (303 str.)
\item tokarski stroj za tokarenje željezničkih kotača (305 str.)
\item tokarski stroj za natražno tokarenje (307 str.)
\item tokarski stroj za obradu koljeničastih osovina (308 str.)
\item tokarski stroj za odrezivanje (309 str.)
\end{itemize}
\end{itemize}
\item Revolverski tokarski stroj - strojevi s više alata na tkz. revolverskoj glavi za mijenjanje alata
\item Poluautomatski tokarski stroj - tu spadaju agregatni strojevi
\item Automatski tokarski strojevi - sa jednim ili više vretena koji mogu biti upravljani numeričkim ili računalnim programom
\end{enumerate}
Tokarski stroj se sastoji od četiti glavna dijela:
\begin{enumerate}
\item Radno vreteno sa sistemom prijenosa od el. motora do radnog vretena, koje se naziva vretenište.
\item Postolje tokarskog stroja sa saonicama po kojima kliže suport i konjić
\item Suporti (obično porečni ili uzdužni) s napravom za stezanje noža, koji preuzimaju pogon od vučnog ili navojnog vretena i ostvaruju posmak
\item Konjić sa šiljkom za učvršćivanje predmeta.
\end{enumerate}
Kod tokarskih strojeva je konjić uvijen s desne strane radnika, a vretenište sa steznom glavom lijevo od radnika, za vrijeme rada.\par
Numerički ili računalno upravljani strojevi uz te dijelove imaju i jedinice za upravljanje strojem. \par
Kada se govori o nekom tokarskom stroju onda se misli na njegove karakteristične podatke a to su:
\begin{enumerate}
\item visina radnog vretena iznad postolja, što definira maksimalni promjer tokarenja
\item razmak vrha glavnog vretena i vrha konjića, što definira maksimalna dužina uzdužnog tokarenja
\item snaga elektromotora s eventualnim stepenicama broja okretaja samog el. motora što je veoma rijetko. Obično je motor s jednim brojem okretaja.
\item broj radnih brzina koje je moguće ostvariti korz vretenište
\end{enumerate}
\subsection{Uređaji za stezanje}
Da bi se predmet obradio na tokarskom stroju potrebno gdje učvrsiti u radno vreteno. Način učvršćivanja ovisi o obliku predmeta koji se obrađuje i o opremljenosti stroju na kojem ćemo predmet obrađivati. Tako raspoznajemo različite načine učvršćivanja stroja.
\subsubsection{Učvršćivanje predmeta između šiljaka radnog vretena i konjića}
%SLIKA
\par
Na kraju predmeta su zabušene posebne vrst provrta za centriranje i predmet je upet između šiljaka. Da bi se osiguralo njegovo zakretanje, na osovinu se montira uređaj koji se zove \textbf{tokarsko srce}. Svojim tlačnim vijkom tokarsko srce se učvrsi na predmet a svojim dužim krajem zapne za izdanak na ploči i na taj način vrti predmet.\par
Postoje dvije vrste šiljaka za takav način učvršćenja predmeta i to:
\begin{itemize}
\item čvrsi šiljci - za veoma fina tokarenja i lakše predmete
\item zakretni šiljci - koji se okreću zajedno s predmetom a pritisak se prenosi preko kugličnih ležajeva.
\end{itemize}
Ukoliko se bušenje predmeta, izrade središnjih provrta za šiljke, izvodi na tokarskom stroju, za pridržavanje predmeta koriste se linete. Postoje takozvane čvrste ili pomične linete sa valjcima.
\subsubsection{Učvršćenje predmeta pomoču čeljusne ploče}
%SLIKA
Na čeljusnoj ploči
 postoje obično 3 ili 4 čeljusti koje se mogu neovisno jedno od druge , sa svojim vijkom za podešavanje, pomicati duž posebnoig kanala u ploči. Ovanj način se koristi kod nesimetričnih oblika koji se moraju tokariti. Postoje uređaji na steznim pločama koji su ugrađeni u stezne ploče, da se prilikom otezanja  jedne čeljusti stežu jednoliko sve tri čeljusti. Takove stezne ploče nazivamo \textbf{amerikanerima}. Danas već postoji čitav niz riješenja amerikanerima. (Horvat 294. - 295. str.)\par
Osim ova dva načina pritezanja, danas postoje i pričvršćenje elektromagnetima i pneumatski ili ihraulično pritezanje.
\subsection{Pričvršćenje noževa}
Danas se učvršćenje noževa izvodi u malim napravicama koje mogu primiti i 4 noža a mogu se zakretati oko svog središta. Te napravice su učvršćene nas suportu stroja.
\subsection{Tokarski noževi} 
\begin{enumerate}
\item Ručni tokarski noževi (Horvat 234.)
\begin{itemize}
\item za glađenje
\item za odrezivanje
\item za urezivanje
\item za unutarnji narez
\item za vanjski narez
\end{itemize}
\item Strojni tokarski noževi - upeti u suportu
\begin{itemize}
\item tokarski noževi od jednog komada
\item tokarski noževi s navarenom oštricom (244. str.)
\item tokarski noževi s držačima oštrice (245. - 248. str.)
\end{itemize}
Tokarske noževe dijelimo na 
\begin{itemize}
\item noževi za vanjsko tokarenje
\item noževi za unutarnje tokarenje
\end{itemize}
Sljedeća podjela noževa je
\begin{itemize}
\item noževi za grubo obrađivanje - Klapstokor nož (52 sl. 70 str.)
\item noževi za fino obrađivanje
\item noževi za postrana obrađivanja (grubo ili fina)
\item noževi za utore i odrezivanja
\item profilni noževi (okrugli fazonski nož sl. 250. str. 240.)
\item noževi za izrađivanje nareza
\item uz odgovarajuće noževe vanjsko tokarenje imamo tako i noževe za unutarnja tokarenja
\end{itemize}
\end{enumerate}
\subsubsection{Režimi i njihovo određivanje}
Režimi ovise o nizu faktora:
\begin{itemize}
\item materijalu kakav obrađujemo
\item da li obrađujemo grubo ili fino
\item s kakvim nožem vršimo obradu.
\end{itemize}
Iz brzine rezanja i promjera obrađivanog predmeta odabiremo broj okretaja tokarskog stroja. Preko mjenjačke kutije, nekada poznati Norton odredimo broj okretaja stroja i veličinu poznatu uz odabranu dubinu rezanja.
\subsubsection{Sile rezanja na tokarskom nožu}
O sili rezanja već smo prije opširno govorili.
\subsubsection{Tokarenje konusa na tokarskom stroju}
Možemo ga izvesti:
\begin{enumerate}
\item Zakretanjem pomoćnog uzdužnog suporta za kut nagiba konusa (Horvat 259. str.)
\item Izbacivanje središnjeg šiljka na konjića iz simetrale stroja
\item Pomoću Kapirne letve (Horvat 283. sl. - 261. str.)
\end{enumerate}
Tokarenje navoja, bilo vanjski ili unutarnji moguće je izvesti za pale proreze pomoću konjića kojim pritiskujemo navojno svrdlo ili utezima (Horvat 263. str. 286 slika), a kod toakrenja velikih nareza svih oblika koristimo kalibrirane noževe. Brušenje centralnih provrta vrši se pomoću konjića u koji se umetne svrdlo.\par
Brušenje vanjsko ili unutarnje vrši se tako da se nasuprot učvrsi odgovarajući nosač brusne ploće i vršsi brušenje. Kod velikih strojeva postoji čitav niz alata i naprava za najrazličitije svrhe:
\begin{itemize}
\item glodanje uzdužnih utora
\item brušenje provrta na prirubnicama
\item brušenje svih vrsta
\item razvrtavanje itd.
\end{itemize}
\subsubsection{Vrijeme obrade na tokarskom stroju}
Kao i svaka norma i ova se sastoji od niza međusobno manje ili više vezanih detalja.
\begin{enumerate}
\item Pripremno vrijeme
\item Čisto strojno vrijeme koje se može odrediti iz odabranih režima rada (brzine rezanja, posmaka i dubine rezanja). A dijeli se na strojno vrijeme, strojno ručno vrijeme i radno vrijeme.
\item Završno vrijeme
\item Dodatno vrijeme
\begin{itemize}
\item Ku - faktor zamora
\item Ka - faktor koj na čovijeka registrira utijecaj okoline
\item Kd - dopunsko vrijeme (propisani odmor, osobne potrebe, organizirani gubitci).
\end{itemize}
Sve to vrijedi za prosječno uviježbanog čovijeka.
\end{enumerate}
\begin{table}[!h]
\centering
\caption{Alati od tvrdih metala}
\label{my-label}
\begin{tabular}{cc|c|c|}
\cline{3-4}
\multicolumn{1}{l}{}                   & \multicolumn{1}{l|}{}                    & \multicolumn{2}{c|}{Brzina rezanja}                                  \\ \hline
\multicolumn{1}{|l|}{Materijal obrade} & \multicolumn{1}{l|}{Čvrstoća materijala} & \multicolumn{1}{l|}{Gruba obrada} & \multicolumn{1}{l|}{Fina obrada} \\ \hline
\multicolumn{1}{|c|}{Če 37}            & 37-42                                    & 180-250                           & 250-350                          \\ \hline
\multicolumn{1}{|c|}{Če 60}            & 60 - 70                                  & 100 - 130                         & 130 - 170                        \\ \hline
\multicolumn{1}{|c|}{Cr-Ni}            & 70 - 85                                  & 80 - 100                          & 100 - 120                        \\ \hline
\multicolumn{1}{|c|}{Nehrđajuči}       & 60 - 70                                  & 40 - 60                           & 60 -90                           \\ \hline
\multicolumn{1}{|c|}{Cr - V -čelici}   & 100                                      & 25 - 45                           & 45 - 80                          \\ \hline
\multicolumn{1}{|c|}{12\% Mn}            &                                          & 10 -25                            & 25 - 40                          \\ \hline
\multicolumn{1}{|c|}{Če - lijev}       & 40 - 50                                  & 90 - 120                          & 120 - 160                        \\ \hline
\end{tabular}
\end{table}
\FloatBarrier
\section{Bušilice}
Bušilice su strojevi, čiji naziv govori o samoj namjeri. One buše provrte u predmetu. Modernije izvedbe bušilica imaju mogučnost narezivanja, razvrtavanja i glodanja. \par
Predmet je redovito čvrsto upet na radnom mjestu, a svrdlo vrši sva druga gibanja, glavno rotiranje oko svoje osi i pomično - aksijalno savladavanje otpora aksijanlog prodiranja u materijal. 
Prva podjela bušilica je:
\begin{itemize}
\item ručne koje unutar sebe možemo podijeliti na :
\begin{itemize}
\item ručni uređaji za bušenje
\item ručni uređaji s pogonom na elek. struju ili komprimirani zrak
\item uređaji za bušenje sa zmijastom osovinom (danas se to koristi za izvijaće)
\end{itemize}
\item strojne koje dalje dijelimo na
\begin{itemize}
\item obične bušilice - za rad sa spiralnim svrdlom koje je učvršćeno u vreteno bušilice
\item bušilice za povečanje postojećeg provrta - koje rade motkama za brušenje s učvršćenim nožom.
\end{itemize}
\end{itemize} 
Prema načinu rada bušilice dijelimo na
\begin{itemize}
\item bušilice s okomitim vretenom koje se dijele na stolne bušilice, konzolne ili radijalne bušilice raznih veličina i viševretene bušilice.
\end{itemize}
Kod bušilica razlikujemo dva osnovna tipa.
%SLIKA
Razlika između ova dva tipa  bušilica je u tome što je stup prve bušilice okrugli a sradni stol se uz vertikalno pomicanje gore dolje može zakretati oko stupa. Te bušilice se nazivaju još i stolnim bušilicama. \par
Druga izvedba bušilica je takova što na stupu ima vodilice za vertikalno pomicanje glave gore dolje, dok je stol stabilani čvrst. \par
Na trećoj skici slaba nam je specijalna izvedba bušilice tkz. \textbf{radijalna bušilica}. Ovaj tip bušilica obično su one večih dimenzija, obično se koriste kod bušenja provrta na velikim predmetima koje je nemoguće premještati pod pinolu svrdla. Ta bušilica sastoji se od postolja, okruglog stupa, u kojem se giba konzola gore-dolje i zakreće oko okruglog stupa, konzole po kojoj se ljevo desno giba nosač alata koji u sebi ima pogonska vretena koje se opet giba gore dolje i vrši glavno gibanje kretanja alata - rotaciju. Uz ovakvu bušilicu ide i postolje koje se može odkloniti. Veličina bušilice definirana je prema največev mogučem svrdlu odnosno prema največoj mogučoj izbušenoj rupi. Stepenovanje bušilica izgleda odprilike ovako:
\begin{center}
6, 10, 16, 25, 32, 40, 50, 63, 80 mm
\end{center}
Posoje DIN standardi kod kojih su definirane neke mjere pa je tako u ovisnosti o maksimalnom promjeru svrdla definirani i drugi podatci a to su:
\begin{itemize}
\item vertikalni hod vretena 80 - 400 mm
\item razmaka od temeljne ploće do vretena do 1320 mm
\item minimalna brzina rezanja kod nazivnoh promjera  35 - 10 m/min
\item minimalan automatski posmak 0,1 - 0,4 mm/okr
\item trajna snaga 0,2 - 12 kW
\end{itemize}
\subsubsection{Viševretene bušilice}
Koriste se kod serijske proizvodnje proizvoda sa više provrta (prirubnice, blokovi motora itd.) Specifičnost prigona je u tome što svako vreteno ima zglobnu vezu s glavnom pogonskom osovinom, ali sama svrdla se mogu postaviti u razne međusobne položaje.
\subsubsection{Koordinatna bušilica}
Za najtočnija obrađivanja. Ovdje se pomiče i stol na kojem je učvršćen predmet. Uz translatorna gibanja može se i zakretati. Nosač alata se također pomiče  



\end{document}