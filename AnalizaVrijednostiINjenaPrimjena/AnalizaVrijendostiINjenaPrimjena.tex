\documentclass[a4paper,12pt]{article}
\usepackage[croatian]{babel}
\usepackage[utf8]{inputenc}
\usepackage[T1]{fontenc}
\usepackage{gensymb}
\usepackage{tikz}
\usepackage{amsmath,amsfonts,amssymb,mathrsfs}
\usepackage{fancyhdr,makeidx}
\usepackage{dcolumn,multirow,eucal,hhline,subcaption}
\usepackage{pgfplots,pgfplotstable,colortbl,array}
\usepackage[unicode, hidelinks]{hyperref}
\usepackage{ragged2e}
\usepackage{
epstopdf,
graphicx,tikz,
%pgflibraryshapes,
color,caption,
listings,
dcolumn,
multirow,
array,
booktabs,
picture,
upgreek,
wrapfig,      
cancel,
placeins,
url,
verbatim,
media9,
float,
incgraph
}
\usepackage[nottoc,numbib]{tocbibind}
\usepackage[croatian]{nomencl}
\makenomenclature
\usepgfplotslibrary{units}
\usetikzlibrary{pgfplots.units}
\usetikzlibrary{angles,calc,decorations.pathmorphing,patterns}
\usetikzlibrary{decorations.pathmorphing,decorations.pathreplacing}
\pgfplotstableset{precision=10,set thousands separator={}}
\usepackage{nicefrac}
\hoffset -30 pt
\voffset -50 pt
\textheight = 650 pt
\textwidth = 450 pt
\sloppy
\definecolor{lightgray}{gray}{0.5}
\setlength{\parindent}{0pt}
\usepgfplotslibrary{external}
\usetikzlibrary{pgfplots.external}
\usepgfplotslibrary{external}
\usepackage{lmodern,textcomp}
\usepackage{multimedia}
\headheight = 14 pt
\headwidth = 17 cm
\pagestyle{fancy}


\newcommand{\tikzAngleOfLine}{\tikz@AngleOfLine} %crtanje kuteva
  \def\tikz@AngleOfLine(#1)(#2)#3{%
  \pgfmathanglebetweenpoints{%
    \pgfpointanchor{#1}{center}}{%
    \pgfpointanchor{#2}{center}}
  \pgfmathsetmacro{#3}{\pgfmathresult}%
  }


\fancyfoot[R]{\thepage}
\fancyhead[L]{Marijan Jasak dipl. ing}
\fancyhead[R]{Analiza vrijednosti i njena primjena}
\fancyfoot[L]{}
\fancypagestyle{fancypage}{
    \fancyhf{}
    \footrulewidth=20pt
    %\renewcommand{\headrulewidth}{0pt}
    \renewcommand{\footrulewidth}{0.4pt}
}
\cfoot[]{}
\renewcommand{\thefigure}{\arabic{section}.\arabic{subsection}.\arabic{figure}}
\makeindex
\numberwithin{figure}{section}

\setlength\parindent{24pt}
\begin{document}

\begin{titlepage}
\begin{titlepage}
  \null\vfill

  \begin{center}

  {\huge\bfseries Analiza vrijednosti \\ i njena primjena}
  \vskip 2cm
  
  \end{center}

\vfill


\begin{tabular}[t]{c}
Marijan Jasak dipl. ing.\\ sour "Rade Končar" \\ Zagreb
\end{tabular}
\hfill
\end{titlepage}
\end{titlepage}
\section*{Analiza vrijednosti i njena primjena}
Živimo u doba naglog razvoja znanosti. Znanosti, koja dovodi do novih materijala i tehnologija, a time i do novih načina proizvodnje. Na taj način se otvaraju fantastične mogučnosti usavršavanja postojećih proizvoda. Dodamo li tome bujnu ljudsku maštu, otvaraju se daljnje perspektive i stvaranja novih proizvoda. Funkcije tih proizvoda ovise o našim, ljudskim, potrebama i intuiciji predlagaća za potrebe sutradašnjice.\par
Danas već možemo proizvesti skoro sve što je netko zamislio. Ali cilj proizvodnje nije u tome da se nešto proizvodipod svaku cijenu, nego:
\begin{center}
POTREBNO JE PROIZVESTI PROIZVOD UZ MINIMALNE TROŠKOVE!
\end{center}
\noindent Tako proizvod treba:
\begin{itemize}
	\item u potpunosti ispuniti funkcije zbog kojih se proizvodi
	\item biti kvalitetno postojan
	\item udovoljiti zahtjevima na tržištu po izgledu savremenosti i količini
	\item biti jeftiniji od konkurentskih proizvoda iste funkcije i kvalitete.
\end{itemize}
\noindent Udovolji li proizvod ove postavke, održat će se na tržištu. Ako su i vlastiti troškovi proizvodnje manji od cijene koju diktira tržište, proizvod stvara i akumulaciju~(Slika \ref{Slika1}). \par
Djelovati na troškove proizvodnje je moguće i na samom početku stvaranja, kako ideje o proizvodu, tako i o načinu proizvodnje. \par
Kod prozivodnje se misli na sve faze razvoja proizvoda počevši od ideje, preko konstrukcije do lansiranja proizvoda na tržište. U tu grupu spada i izrada tehničko ekonomskih alaniza postoječih proizvoda za iznalaženje najboljeg proizvodng nivoa. \par
Kod proizvodnje se misli na sve faze eventualnog projektiranja i izgradje nove tvornice, koja će proizvoditi predviđeni proizvod, odnosno, otkrivati mogučnosti smanjenja vlastitih troškova proizvodnje u postoječoj proizvodnji. Traženje najpovoljnijeg riješenja proizvodnje nekog proizvoda nije mali i jednostavan zadatak. To je kompleksni problem, koji se praktiči mjenja iz dana u dan. Jer, ONO ŠTO JE JUČER BILO NAJBOLJE, DANAS JE JOŠ DOBRO, A SUTRA VIŠE NEĆE ZADOVOLJAVATI!

\begin{figure}
\centering
\input{formiranje_cijene} 
\caption{Formiranje cijene proizvoda na tržištu}\label{Slika1}
\end{figure}




\end{document}